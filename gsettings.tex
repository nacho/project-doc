% mainfile:main.tex
% revised by dnolivieri

\chapter{GSettings port}\label{chap:GSettings}

In section \ref{sec:GSettings}, it was explained that GSettings is part of glib and provides a convenient API for storing and retrieving application settings.

Until now, gedit was using GConf for storing its settings. GConf has been \emph{deprecated} for this cycle of \GNOME, meaning that all \GNOME applications have to be ported to GSettings if they want to still be part of the core applications for the \GNOME OS.

An early adoption of this library was decided to be done in order to push the development and have it ready for \GNOME 3.0. This meant a lot of API changes, having a development branch for the port, several design mistakes and a lot of extra work.   The good thing is that it helped to have a good API, a good library design, and helped to solve several problems in GSettings.

\section{Analysis and Design}

\subsection{Previous design}

The way gedit had implemented the settings system  was probably one of the oldest parts of the gedit code base. It was implemented in pure C, without the Object Oriented paradigm. The reason it was not changed before was that even if you are using an iterative design, the code was working perfectly, it was `easy' to maintain and it did not have important bugs.

\newpage
\subsubsection*{gedit-prefs-manager-app}

This file had two main methods: \textit{gedit\_prefs\_manager\_app\_init} and \textit{gedit\_prefs\_manager\_app \_shutdown}. This methods where called in the initialization of the program and in the finalization respectively.

The \emph{init} method took care of initializing the system in order to listen to changes in the settings. This was possible to some signals provided by GConf. Once some change were done in some of the settings, the prefs-manager-app obtained the changed signal and changed the respective setting to the UI of the application.

The \emph{shudown} method took care of finalizing all the resources taken by the init method, in order to not leak memory at the end of the program and avoid crashes due to those resources.

\subsubsection*{gedit-prefs-manager}

This file provided  a set of methods, three per setting: get/set/can\_set. Also a default value was provided for each setting in case this setting was not set yet. (This was one of the deficiencies of GConf, it will be explained later).

In order to achieve these methods, a set of macros working as templates were provided, in the way that calling a macro with a name was creating the respective implementation of the method, leading to only create apart from the macro call, the semi-public prototype for this method. An example of this is shown below:

\begin{lstlisting}[style=GObject]

/* .c file */

/* Macro for boolean settings */
#define DEFINE_BOOL_PREF(name, key, def) gboolean 	\
gedit_prefs_manager_get_ ## name (void)			\
{							\
	gedit_debug (DEBUG_PREFS);			\
							\
	return gedit_prefs_manager_get_bool (key,	\
					     (def));	\
}							\

DEFINE_BOOL_PREF (insert_spaces,
		  GPM_INSERT_SPACES,
		  GPM_DEFAULT_INSERT_SPACES)

/* .h file */

#define GPM_DEFAULT_INSERT_SPACES	0 /* FALSE */

/* Insert spaces */
gboolean gedit_prefs_manager_get_insert_spaces (void);
void	 gedit_prefs_manager_set_insert_spaces (gboolean ai);
gboolean gedit_prefs_manager_insert_spaces_can_set (void);

\end{lstlisting}

\subsection{New design, Part 1}

First of all,  it must be understood how GSettings was suppose to work in order to make any decisions in the design. GSettings was based mainly on schemas. A schema is a document that specifies all the settings for an application, with the type of the setting, the default value and some description for it. A schema file would look like this:

\begin{lstlisting}[style=GObject]
schema org.gnome.gedit:
  path /apps/gedit/

  child preferences:
    child editor:
      key use-default-font = @b true
      key editor-font = @s 'Monospace 12'
      key scheme = @s 'classic'

\end{lstlisting}

As we can see a schema is defined with a specific namespace. For \GNOME applications there was a convention to start it by \emph{org.gnome}. Apart from the namespace we had as well:
\begin{itemize}
  \item \textbf{path}: The path the schema points to.
  \item \textbf{child}: It was a way to sort and manage easily the settings.
  \item \textbf{key}: The specific key for the setting. Only dashes were allowed for names to separate words.
\end{itemize}

In order to access the settings,  one had to access the main schema and then iterate over the children. For example, if we wanted to get the value of use-default-font we needed to do something like this:

\begin{lstlisting}[style=GObject]
settings = g_settings_new ("org.gnome.gedit");
prefs_settings = g_settings_get_child (settings, "preferences");
editor_settings = g_settings_get_child (prefs_settings, "editor");

g_settings_get (editor_settings, "use-default-font", "b",
                &default_font);
\end{lstlisting}

This leaded to the first design. As can be visualized in the figure \ref{fig:GSettingsFirst}, a new object \emph{GeditSettings} will inherit from a GSettings object and the application will manage the life of this object.

\addfigure[scale=0.75]{./images/gsettings1}{GSettings first design}{fig:GSettingsFirst}

On construction,  GeditSettings will point to the root of the schema by setting the \emph{schema} property on construction time to `org.gnome.gedit'.
In order to make it easier for the user, a new API had to be added to manage GeditSettings as shown below. This API will take care of getting the specific child object.

\begin{lstlisting}[style=GObject]
GSettings *
gedit_app_get_settings (GeditApp    *app,
			const gchar *path_list,
			...)
{
	GSettings   *settings;
	va_list      args;
	const gchar *path;

	g_return_val_if_fail (GEDIT_IS_APP (app), NULL);

	settings = app->priv->settings;
	g_object_ref (settings);

	va_start (args, path_list);
	for (path = path_list; path; path = va_arg (args, const gchar *))
	{
		GSettings *aux;

		aux = g_settings_get_child (settings, path);
		g_object_unref (settings);
		settings = aux;

		if (settings == NULL)
		{
			va_end (args);
			return NULL;
		}
	}
	va_end (args);

	return settings;
}
\end{lstlisting}

In relation to the implementation of GeditSettings, this object will make something similar to what it was doing gedit\_prefs\_manager\_app, listen to the changes in the settings database and set the relative settings to the UI of the application. For this, it will listen to the `changed' signal for the GSettings object. An advantage that provides GSettings over GConf is that it allows the application to listen to the changes for a specific setting instead of the whole directory of settings (child in terms of gsettings), making it easier to implement and set the ui changes.

At this point of the development, gedit was the only big application ported to GSettings. GSettings was getting quite ready, so it was decided to make a hackfest taking gedit as an example to see where it could be improved API wise to make it easier for application developers. The information about this hackfest can be found in the next url: \url{https://live.gnome.org/Hackfests/GSettings2010}

\newpage
\subsection{New design, Part 2}

After the hackfest, some changes were made in GSettings that allowed to simplify our current implementation. First, the main important change was to not needed anymore to iterate over children. In the figure \ref{fig:GSettingsSecond} it is shown how this affects our previous design.

\addfigure[scale=0.75]{./images/gsettings2}{GSettings second design}{fig:GSettingsSecond}

We can see that we are not inheriting anymore from GSettings but from GObject. The fact that we can make direct access to the settings, provided us the fact to just have a listener object that listens the settings changes and do not need to be accessed by any other object. This means also, that the API exposed in GeditApp \emph{gedit\_app\_get\_settings} could be removed. Below we can see how the way to access any settings has changed:

\begin{lstlisting}[style=GObject]
/* BEFORE */
settings = g_settings_new ("org.gnome.gedit");
prefs_settings = g_settings_get_child (settings, "preferences");
editor_settings = g_settings_get_child (prefs_settings, "editor");

/* AFTER */
settings = g_settings_new ("org.gnome.gedit.preferences.editor");
\end{lstlisting}

Apart from the change in the design, the fact that to access any setting, it was needed to use g\_settings\_get, it was not very handy for application users. In this hackfest, it was decided to be added easier to use API for getters and setters of the most important settings types. Below it is shown an example for this:

\begin{lstlisting}[style=GObject]
/* BEFORE */
g_settings_get (editor_settings, "use-default-font", "b",
                &default_font);

/* AFTER */
default_font = g_settings_get_boolean (settings, "user-default-font");
\end{lstlisting}

The last important change made after this hackfest, was the change in the way an schema file is defined. The main reason for this, is that the previous format was not an standard. It was pointless to try to get a new format and a new standard for files when internally this format was converted to \emph{XML} and this language is already an standard and known by a lot of programmers already. Thus, in the end it was decided to use XML directly. In the example below,  the same schema as berfore is shown it was shown, yet in XML:

\begin{lstlisting}[style=xml]
<schemalist>
  <schema gettext-domain="@GETTEXT_PACKAGE@" id="org.gnome.gedit" path="/org/gnome/gedit/">
    <child name="preferences" schema="org.gnome.gedit.preferences"/>
    <child name="state" schema="org.gnome.gedit.state"/>
    <child name="plugins" schema="org.gnome.gedit.plugins"/>
  </schema>
  <schema gettext-domain="@GETTEXT_PACKAGE@" id="org.gnome.gedit.preferences" path="/org/gnome/gedit/preferences/">
    <child name="editor" schema="org.gnome.gedit.preferences.editor"/>
    <child name="ui" schema="org.gnome.gedit.preferences.ui"/>
    <child name="print" schema="org.gnome.gedit.preferences.print"/>
    <child name="encodings" schema="org.gnome.gedit.preferences.encodings"/>
  </schema>
  <schema gettext-domain="@GETTEXT_PACKAGE@" id="org.gnome.gedit.preferences.editor" path="/org/gnome/gedit/preferences/editor/">
    <key name="use-default-font" type="b">
      <default>true</default>
      <_summary>Use Default Font</_summary>
      <_description>Whether to use the system's default fixed width font for editing text instead of a font specific to gedit. If this option is turned off, then the font named in the "Editor Font" option will be used instead of the system font.</_description>
    </key>
    ...
    ...
  </schema>
</schemalist>
\end{lstlisting}

\section{Implementation}

The  implemention followed directly the approach described above.  The patches 
can be found below in the bugs section. Also, since it was quite a big change, it was developed in a different 
branch until it was ready to be merged. The branch can be found in: \url{http://git.gnome.org/browse/gedit/log/?h=gsettings}.

\subsection{Bugs}

\noindent\url{https://bugzilla.gnome.org/show_bug.cgi?id=618100}\\
\noindent\url{https://bugzilla.gnome.org/show_bug.cgi?id=618144}\\
\noindent\url{https://bugzilla.gnome.org/show_bug.cgi?id=618159}\\
\noindent\url{https://bugzilla.gnome.org/show_bug.cgi?id=618712}\\
\noindent\url{https://bugzilla.gnome.org/show_bug.cgi?id=620554}\\
\noindent\url{https://bugzilla.gnome.org/show_bug.cgi?id=618523}\\
\noindent\url{https://bugzilla.gnome.org/show_bug.cgi?id=623220}\\
\noindent\url{https://bugzilla.gnome.org/show_bug.cgi?id=619898}\\
\noindent\url{https://bugzilla.gnome.org/show_bug.cgi?id=625498}\\
\noindent\url{https://bugzilla.gnome.org/show_bug.cgi?id=649717}

\section{Tests}

To test this implementation, it had to be key by key, enabling and disabling and checking that all the setting were set correctly.
