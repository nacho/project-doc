% mainfile:main.tex
% revised by dnolivieri

\chapter{Introduction}


This goals of this Final Year Project (for the Engineering Degree at ESEI) has been 
concerned with adding several profoundly new features and internal software upgrades 
to the well known \GNOME application, \emph{gedit}.  In particular, the work 
described pertains to a major upgrade from the stable version of \emph{gedit} (version 2.30) 
to version 3.0, representing a significant technology contribution 
(see page \pageref{chap:Technologies}).   All work described in this Final year thesis project 
has been performed by the author, while keeping with established \GNOME schedules.


Since of the beginning of the \GNOME Project, \emph{gedit} has been the \GNOME default basic 
text editor. The \GNOME Project was started in 1997 by two university students, Miguel de Icaza and 
Federico Mena. Their aim was to produce a free (as in freedom) desktop environment. Since 
then, \GNOME has grown into a hugely successful enterprise. Used by millions of people 
across the world, it is the most popular desktop environment for GNU/Linux and UNIX-type 
operating systems. The desktop has been utilised in successful, large-scale enterprise and 
public deployments, and the project’s developer technologies are utilised in a large number 
of popular mobile devices\cite{website:gnome}. 


\addfigure[scale=0.41]{./images/gedit}{Gedit window}{fig:Gedit}


While the desktop environment has evolved and matured, the \emph{gedit} application has 
not scaled at the same pace, nor has gained modern sophisticated features expected of 
modern editors.   This project attempts to amend this situation by endowing \emph{gedit} 
with modern and rich features, shared by most editing applications today, as well 
as upgrade underlying dependencies on modern libraries, in order to plan for future 
scaling and powerful plugin interoperability.    Thus, the project 
attempts to perform a major revisional change, while adhering to the open-source community 
standards and timelines. 

















