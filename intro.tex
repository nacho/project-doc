% mainfile:main.tex
% revised by dnolivieri


\chapter{Introduction}

The \GNOME Project was started in 1997 by two university students, Miguel de Icaza and Federico Mena. Their aim was to produce a free (as in freedom) desktop environment. Since then, \GNOME has grown into a hugely successful enterprise. Used by millions of people across the world, it is the most popular desktop environment for GNU/Linux and UNIX-type operating systems. The desktop has been utilised in successful, large-scale enterprise and public deployments, and the project’s developer technologies are utilised in a large number of popular mobile devices\cite{website:gnome}.

\addfigure[scale=0.41]{./images/gedit}{Gedit window}{fig:Gedit}

Since the project started, \emph{gedit} has been the \GNOME default text editor. As such, this program must be up to date and must evolve with the desktop. The \GNOME community defines the schedules between versions for each program and also sets which libraries fit the standards to be used by the applications, or which libraries should get deprecated.

This goals of this project is to add proposed new features to \emph{gedit}
as well as using these new technologies (see page \pageref{chap:Technologies}) 
in the major upgrade from versions 2.30 to version 3.0. 




