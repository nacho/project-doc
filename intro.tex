% mainfile:main.tex

\chapter{Introduction}

The GNOME Project was started in 1997 by two then university students, Miguel de Icaza and Federico Mena. Their aim: to produce a free (as in freedom) desktop environment. Since then, GNOME has grown into a hugely successful enterprise. Used by millions of people across the world, it is the most popular desktop environment for GNU/Linux and UNIX-type operating systems. The desktop has been utilised in successful, large-scale enterprise and public deployments, and the project’s developer technologies are utilised in a large number of popular mobile devices\footnote{\url{http://www.gnome.org/about/}}.

Since the project has started, \emph{gedit} has been the default text editor. As part of GNOME this program must be up to date, evolving with the desktop. The GNOME comunity is the one that defines the schedules between versions for each program and also sets which libraries fit the standards to be used by the applications, or which libraries should get deprecated.

With this project it is proposed to add new features to \emph{gedit} and develop it from the version 2.30 to the version 3.0 in the way of using this new technologies (see page \pageref{chap:Technologies}).
