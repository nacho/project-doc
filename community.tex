% mainfile:main.tex

\chapter{Community}

%FIXME: first draft, mainly taken from gedit.org, needs better explanations

When you work in a free software project managed by the community, it is very important to be in contact with them. In this chapter it will be explained how to get in touch with these people and how to do to contribute with a specific project, in this case a focus will be put on \emph{gedit}\footnote{\url{http://www.gedit.org}}.

\section{Contributing}\label{Contributing}

gedit development relies on voluntary contributions and everyone is invited to help.

If you are interested in helping us to develop gedit, please contact the developers on IRC and/or send a message to the gedit mailing list and offer your help.

\section{Code Repository}\label{Repository}

gedit is hosted on the \emph{GNOME Git repository}\footnote{\url{http://git.gnome.org/}} in the gedit module\footnote{\url{http://git.gnome.org/browse/gedit}}.

\section{Bug reporting}\label{Bugs}

Bugs should be reported to the \emph{GNOME bug tracking system}\footnote{\url{http://bugzilla.gnome.org/}}, product gedit\footnote{\url{http://bugzilla.gnome.org/browse.cgi?product=gedit}}.

You will need an email address to register (if you haven't already) before you can use the system to file a new bug report.

There are guidelines\footnote{\url{http://bugzilla.gnome.org/page.cgi?id=bug-writing.html}} for reporting new bugs, try to make sure you follow them, it helps us help you much more effectively. The use of keywords\footnote{\url{http://bugzilla.gnome.org/describekeywords.cgi}} also helps us prioritize bugs.

\section{Patches}\label{Patches}

Patches should be submitted to the bug tracking system (see \ref{Bugs}).

If the patch fixes a known bug, add the patch as an attachment to the corresponding bug report. Otherwise, enter a new bug report that describes the problem the patch fixes, and attach it to that bug report.

Patches should be generated using the git format-patch command and should include a descriptive commit message. See the HACKING\footnote{\url{http://git.gnome.org/cgit/gedit/tree/HACKING}} file for more detailed guidelines.

\section{Mailing lists}\label{Mailing}

You can subscribe to the \emph{gedit mailing list}, or change your existing subscription, visiting the gedit-list info page\footnote{\url{http://mail.gnome.org/mailman/listinfo/gedit-list}}.

To post a message to all the list members, send an email to \href{mailto:gedit-list@gnome org}{gedit-list (at) gnome org}.

To see the collection of prior postings to the list, visit the \href{http://mail.gnome.org/archives/gedit-list/}{gedit-list archives}\footnote{\url{http://mail.gnome.org/archives/gedit-list/}}.

\section{Internet Relay Chat}\label{IRC}

You can find the developers on \#gedit channel at irc.gnome.org.

\section{Wiki}\label{Wiki}

We have a Wiki for drafting, designing and proposing gedit development available at \url{http://live.gnome.org/Gedit}.
