% mainfile:main.tex

\chapter{Community}

When you work in a free software project managed by the community, it is very important to be in contact with them. In this chapter it will be explained how to get in touch with these people and how to do to contribute with a specific project, in this case a focus will be put on \emph{gedit}\footnote{\url{http://www.gedit.org}}.

\section{Contributing}\label{Contributing}

gedit development, as many other projects from GNOME, relies on voluntary contributions and everyone is invited to help.

If anybody is interested in helping to develop gedit, the main way for contacting the developers is by IRC (see \ref{IRC}) or sending a message to the gedit mailing list (see \ref{Mailing}).

\section{Code Repository}\label{Repository}

gedit is hosted on the \emph{GNOME Git repository}\footnote{\url{http://git.gnome.org/}} in the gedit module\footnote{\url{http://git.gnome.org/browse/gedit}}. To be able to push the changes to this repository, it is needed a especial account that has to be requested to the \emph{GNOME} accounts team. To get it a reasonable number of patches or bugzilla reports will have to be submitted. See \ref{Bugs} and \ref{Patches}.

\section{Bug reporting}\label{Bugs}

Bugs should be reported to the \emph{GNOME bug tracking system}\footnote{\url{http://bugzilla.gnome.org/}}, product gedit\footnote{\url{http://bugzilla.gnome.org/browse.cgi?product=gedit}}.

It is needed an email address to register before the system can be used to file a new bug report.

There are guidelines\footnote{\url{http://bugzilla.gnome.org/page.cgi?id=bug-writing.html}} for reporting new bugs, if they are followed, it will help to make things more effectively. The use of keywords\footnote{\url{http://bugzilla.gnome.org/describekeywords.cgi}} also helps us prioritize bugs.

\section{Patches}\label{Patches}

Patches should be submitted to the bug tracking system (see \ref{Bugs}).

If the patch fixes a known bug, it has to be added as an attachment to the corresponding bug report. Otherwise, it should be created a new bug report that describes the problem the patch fixes, and attach it to that bug report.

Patches should be generated using the \textit{git format-patch}\footnote{\url{http://www.kernel.org/pub/software/scm/git/docs/git-format-patch.html}} command and should include a descriptive commit message. See the \textbf{HACKING}\footnote{\url{http://git.gnome.org/cgit/gedit/tree/HACKING}} file for more detailed guidelines.

\section{Mailing lists}\label{Mailing}

You can subscribe to the \emph{gedit mailing list}, or change your existing subscription, visiting the gedit-list info page\footnote{\url{http://mail.gnome.org/mailman/listinfo/gedit-list}}.

To post a message to all the list members, send an email to \href{mailto:gedit-list@gnome org}{gedit-list (at) gnome org}.

To see the collection of prior postings to the list, the \href{http://mail.gnome.org/archives/gedit-list/}{gedit-list archives}\footnote{\url{http://mail.gnome.org/archives/gedit-list/}} should be visited.

\section{Internet Relay Chat}\label{IRC}

The developers can be found on \#gedit channel at \url{irc.gnome.org}.

\section{Wiki}\label{Wiki}

A Wiki is provided for drafting, designing and proposing gedit development, available at \url{http://live.gnome.org/Gedit}.
