% mainfile:main.tex
% revised by dnolivieri


\chapter{GTK+ 3.0}

%FIXME: draft

\section{API changes}

As explained in other previous chapters, one of the main problems of this release, is that in order to obtain a good GTK+ version, there were a lot of API changes and removals.

\subsection{GSEAL}

GSEAL is a macro that seals public structure fields. The reason for this is that public structure fields are really bad 
in order to improve internal things. A field should only be accessed with getters and setters, so in case the internal 
implementation is wrong this fields can be modified. The reason for GSEAL is to give a compilation warning if an application 
is using any of those public fields.

gedit used several of this public fields, since there was no way to access those data from a getter of a setter method. 
The gedit team provided several patches to GTK+ in order to get methods to access those fields for GtkTextView main classes 
in order to be able to get this work done as fast as possible.

\subsection{Removal of GSEALs}

The GSEAL work was done as a provisional work for gtk+ 2.24.  In this way,  applications had warnings indicating that this field 
would be removed for the 3.0 version.

What was done in this case was to move these sealed fields to a private struct only accessible by the widget itself. 
This was a time intensive job. In my case, I helped with the GtkTextView classes.

\subsection{Direct use of cairo}

Already for several years, Cairo had been the main drawing library in GTK+.  Nonetheless, the truth was that 
within GTK+, there was a complicated mix of legacy drawing systems working together with Cairo. 
To simpllify this, in GTK+3, it was decided to remove this abstraction layer in order 
to make the code easier to maintain and use.  Apart from making an easier to use and to learn API, 
it would make the drawing system a faster, since it facilitate direct calls to the underlying cairo system.

\section{CSS}

One of the new main features for GTK+, apart from all the cleanup made in order to get the 3.0 version ready, was the 
addition of CSS like themes.

The main problem here, was that GTK+ used a theme system that was difficult to maintain and to create new themes. CSS is one of the 
most standard technologies at present for performing this type of task. It is used everywhere on the web and is known by most designers. 

With respect to the gedit implementation, this posed a problem since several custom widgets had been designed to give a 
richer look.  One example was the custum implementation of the button widget in the tab had been written to make the button appear 
smaller than the standard button.

gedit followed this development closely and changed the pertinent widgets with the added API in order to control the 
CSS style with the widgets. The main problem with these changes was the lack of documentation. The main development was made with trial 
and error,  and for some time parts of the gedit's widgets had an unpleasant appearance since there was no way to get the same results 
as with the previous theme system. This resulted in many several bug entries and the need to fix each one.

Apart from gedit, we had to port GtkSourceView to the new theme system. GtkSourceView provides a way to select a specific theme for the viewer, to have different colors for highlighting the text. The problem  was to set the colors for the gutter and to set the view color.

After some discussion with the GTK+ maintainers, the problem was resolved by providing a new CSS class called `view'. 
This class was then used in GtkTextView to set the main view color.

\begin{lstlisting}[style=plain]
GtkTextView {
  color : red;
  background-color : blue;
}

GtkTextView.view {
  color: black;
  background-color : white;
}
\end{lstlisting}

This CSS snippet would make the gutter of the Text view with blue background and red text color, and the background color of the 
main view white with the text color black.

This fulfilled our GtkSourceView requirement to be able to have custom user themes.

The patch I provided to GtkTextView was quite trivial in this case as it can be seen in the snippet below, adding the .view class to the default theme of GtkTextView.

\begin{lstlisting}[style=plain]

index 6f72071..02c3fff 100644
--- a/gtk/gtkcssprovider.c
+++ b/gtk/gtkcssprovider.c
@@ -3638,7 +3638,12 @@ gtk_css_provider_get_default (void)
         "  color: shade (@bg_color, 0.7);\n"
         "}\n"
         "\n"
-        "GtkTreeView, GtkIconView, GtkTextView {\n"
+        "GtkTreeView, GtkIconView {\n"
+        "  background-color: @base_color;\n"
+        "  color: @text_color;\n"
+        "}\n"
+        "\n"
+        ".view {\n"
         "  background-color: @base_color;\n"
         "  color: @text_color;\n"
         "}\n"

\end{lstlisting}

\newpage
\section{Case insensitive search}

Given that features in 3.0 could break the API, it was decided to move the case insensitive search from GtkSourceView 
to GtkTextView directly. This was just an extensible of the GtkTextView search implementation, so it was very easy to move it to GtkTextView.

The main problem with this change  was that GtkTextView did not have any unit test for its current implemented search, 
dealing to create them and test it.   With this, it was detected a bug in the search system where backward search was 
not working correctly. This bug was there for almost ten years without being detected, the reason for this was that 
almost everybody was using the GtkSourceView search instead of the GtkTextView one. This shows the importance of the unit tests.

Once we had everything working correctly, the extra code was added with the specific unit tests to be able to make 
case insensitive searches.

\subsection{Bug}

\noindent\url{https://bugzilla.gnome.org/show_bug.cgi?id=61852}
