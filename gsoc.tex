% mainfile:main.tex
% revised by dnolivieri
\chapter{Google summer of code}

\emph{Google Summer of Code} is a global program that offers student developers stipends to write code for various open source software projects.\cite{website:soc}

Some of the features and bug fixes made in this project have grown out my participation in the \emph{Google summer of code 2010}. Here, a brief description of the steps for applying for this Google \emph{internship} is presented together with as short description of what I have accomplished within this program.

\addfigure[width=0.5\textwidth]{./images/gsoc}{GSoC 2010 Logo}{fig:GSoCLogo}

\newpage
\section{Proposing a good project}\label{sec:GoodProject}

There are several ways to propose a project for the Google summer of code. The main ones are:
\begin{itemize}
  \item Look for a free software application that you find appealing, and propose an improvement.
  \item Talk with the application maintainers and ask them for interesting projects that they would like to mentor. This is probably the most standard and best methods if you want to get your project \emph{accepted}.
\end{itemize}


The code projects are selected from the the Goolge accepted organizations.   A wiki is provided that posts some proposals for students.  As an example, the set of projects proposed by \GNOME for 2010 can be seen here: \url{https://live.gnome.org/SummerOfCode2010}.

\section{Creating the application form}\label{ApplicationForm}

The proposed project description should be supplied in the application form together with a detailed 
schedule of the tasks to be performed on a weekly basis.   The \emph{application form}  used for this project 
is given below. 
\subsection*{gedit improvements}

\subsubsection*{gedit split view}

At present, gedit does not have the ability to provide several views for the same document. I would like to change the internal design of gedit in order to provide this feature. The idea is that a document can be splitted horizontally or vertically depending on what the user prefers. For doing this some menu items will be provided.  Due to the current design of GtkTextView,  this is not an easy task as we need to load the document again every time we split it, and keep it sync.

\paragraph{Why is this important?}

This is an important features since it can allow users to see several parts of the same document at the same time, which is necessary when you are editing long files.

\subsubsection*{gedit tab groups}

Split view is a really good thing but does not provide a way to see different documents at the same time.

By tab groups we refer to the ability of grouping gedit tabs in different groups that can be viewed at the same time in a gedit window.
\begin{itemize}
  \item By default a gedit window has a single tab group (current behavior).
  \item The same document can NOT appear in different tab groups or different tabs.
\end{itemize}

\paragraph{Why is this important?}

It is important to compare different files side by side.

\subsubsection{gedit multi-views}

Apart from split view and tab groups, I'd like to add support (if time permits) for multi views. This means to add abstraction for the document and for the text viewer. In this way we could be able to open a binary file with its own viewer. Or watch a specific html file using webkit as the viewer.\footnote{\url{http://live.gnome.org/Gedit/Multiviews}}

\subsubsection{Timeline}

A proposed timeline (but highly flexible depending on the time needed):

\begin{itemize}
  \item Week 1 and 2: Provide a patch for being able to create tab groups.
  \item Week 3 and 4: See if we need to refactor code to maintain easier the tab groups and start with the split view patch.
  \item Week 5 and 6: Finish the split view patch and see if it need some refactor.
  \item Week 7 and 8: Start the work on the gedit view interface.
  \item Week 9 and 10: Finish the gedit view abstraction and see if worth start working the document abstraction or just bug fixing.
  \item Week 10 and 11: Finish the document abstraction if possible or make stability work.
\end{itemize}

\subsubsection{About me}

My name is Ignacio Casal Quinteiro, I'm 23 years old and I'm studying computer engineer at the Vigo University of Spain. I have been contributing to \GNOME for several years, from tasks like coordinating the Galician translation team to applications like gedit or gtranslator. I love free software and I love \GNOME, the community around it is just awesome, and I would really like to continue my contributions on making it better than it is right now. Even if I already contributed to \GNOME, this project would be a major rewrite of the gedit internals, and the amount of work involved is proportional to a good GSoC. Also making this GSoC would give me the opportunity to spend the summer working on \GNOME instead of another part time job.

\section{Reporting the work}\label{ReportingGSoC}

Depending on the organization, it will be different the way to report what it has been done every week. In the case of \GNOME it is mandatory to make three things:

\begin{itemize}
  \item Create a portfolio: this is a document in the wiki of \GNOME where it has to be explained what you project is about, where to find the code and the links to the documentation reported. For an example here is my portfolio: \url{https://live.gnome.org/SummerOfCode2010/IgnacioCasal_gedit}.
  \item Mail and blog post: every week it has to be sent a post and an email explaining what it has been done, what it will be done next week and if the timeline is under schedule. See one blog post \url{http://blogs.gnome.org/nacho/2010/07/25/gsoc-weekly-report-9/}.
\end{itemize}
