% mainfile:main.tex

\chapter{New search system}

\section{Analysis and Design}

\addfigure[scale=0.75]{./images/gedit-old-find}{Old search dialog}{fig:OldFind}

As we can see in the figure \ref{fig:OldFind}, gedit's search dialog has not evolved over the past years. This way for presenting to the user a dialog for searching has a really big problem, it obscures the text you are searching for. In the past few years a few applications like Firefox or Chrome have come with with some ideas, which are the ones used right now by most software applications for searching in the documents.

\newpage
\subsection{Firefox vs Chrome}

In the figure \ref{fig:FirefoxSearch}, it is shown the method Firefox uses for prompting the search to the user. This method has a few problems.
\begin{itemize}
  \item Some studies showed that presenting visual elements to the user in the bottom of a screen are less discoverable than elements in the top-left of the screen.
  \item It resizes the web browser viewer to present the search widget. This also has an advantage, it will not obscure any part of the document.
  \item It takes a big part of horizontal space.
\end{itemize}

\addfigure[scale=0.5]{./images/firefox-search}{Firefox search}{fig:FirefoxSearch}

As shown in the figure \ref{fig:ChromeSearch}, Chrome comes with an idea based on the Firefox search, but trying to fix some of the problems introduced by the previous one. Although this ones is not perfect either.
\begin{itemize}
  \item It shows the pop up in the top-right, which is not the top-left as the studies demonstrated. Although, this is still better than in the bottom of the screen.
  \item It obscures part of the document.
\end{itemize}

\addfigure[scale=0.5]{./images/chrome-search}{Chrome search}{fig:ChromeSearch}

In the analysis of the currently used search systems, we can see that there is no perfect system to manage a search by the user. In gedit, after some discussion, we decided that the best option for us would be a Chrome like search widget, as at the end it is not like people are writing that much in a text editor horizontally but doing it vertically. This means that poping out a widget on the top-right of the screen would not be that bad for the user. Apart from this problem, it was the less problematic search way for our understanding.

\subsection{Current design}

Below it is pointed out in the figure \ref{fig:OldClassDiagram}, how it looks the current design before introducing anything new, in order to see how things will evolve.

\addfigure[width=\textwidth]{./images/search-system-old}{Old class diagram}{fig:OldClassDiagram}

\subsection{New design}

In the figure \ref{fig:SearchClassDiagram} is spotted the new design in order to implement the new search pop up. The main problem here is that we do not have a container which will allows us to place a widget in a fixed position having a main widget below it. The only widget that allows fixed positions in GTK+ is \emph{GtkFixed}, the problem of this widget is that it does not allow a main widget design. Here is where we need to get the requisites for this widget:

\subsubsection{GeditOverlay}

GeditOverlay is a container which inherits from GtkContainer as any other implemented in the GTK+ toolkit. This container will need:
\begin{itemize}
  \item A main widget, which will allocate the whole space that can count with.
  \item A way to add widget overlaying the main widget in fixed positions. For this we can use the vertical and horizontal alignments provided by GtkWidget.
\begin{lstlisting}[style=GObject]
typedef enum
{
  GTK_ALIGN_FILL,
  GTK_ALIGN_START,
  GTK_ALIGN_END,
  GTK_ALIGN_CENTER
} GtkAlign;
\end{lstlisting}
  \item A way to set an offset in case we do not want the added widget in the edge of the main widget. I.e like in the figure \ref{fig:ChromeSearch} where there is an extra space in the right.
  \item A way to animate a specific widget. See \ref{sec:Theatrics} for an explanation about this.
\end{itemize}

This widget does not add any specific API apart from gedit\_overlay\_add.

\begin{lstlisting}[style=GObject]
void	 gedit_overlay_add (GeditOverlay             *overlay,
			    GtkWidget                *widget,
			    GeditOverlayChildPosition position,
			    guint                     offset);
\end{lstlisting}

\subsubsection{GeditOverlayChild}

This is widget is needed by GeditOverlay to wrap widgets added to it. The reason for this is that all widgets added in GeditOverlay need to be windowed and widgets like GtkEntry or GtkLabel are not windowed. So the main target for GeditOverlayChild is to create a window and set the added widget to this window.

\subsubsection{GeditFloatingSlider}

Inherits from GeditOverlayChild. This widget will be wrapped as a GeditTheatricsActor and in each iteration set its window size to the percent of the animation. In this way we will have a slide in or out effect depending on the moment of the animation.

\subsubsection{GeditRoundedFrame}

GTK+ GtkFrame does not draw a rounded frame in bottom. This widget is more like a stylish way for making prettier the search widget and does not provides any specific capabilities except the stylistic one.

\subsubsection{GeditViewFrame}

GeditViewFrame is a delegate widget in order to remove complexity from GeditTab. This widget will take care of creating the Overlay, the View and the Document and add them correctly.

Apart from that it will also take care of managing the search widget, setting when it must do a specific animation and provide API to invoke it from the GeditWindow's menu.

The API provided by this widget is:

\begin{table}[H]
  \begin{center}
    \begin{tabularx}{\textwidth}{|X|X|}
      \firsthline
      \textbf{Method name:} & \textbf{Explanation:} \\
      \hline
      get\_document & Gets the GeditDocument. This is an utility method to not have to get the view and get the buffer from the view. \\
      \hline
      get\_view & Gets the GeditView. \\
      \hline
      popup\_search & Shows the search pop up. \\
      \hline
      clear\_search & Removes any highlighted occurrences in the text. \\
      \lasthline
    \end{tabularx}
    \caption{GeditViewFrame methods explanation}
  \end{center}
\end{table}

\subsubsection{GeditSearchWidget}

This widget is a simple box container, it adds the entry and the buttons to go to the next occurrence or to the previous one. It will listen to the changes on the entry text and search in the document as appropriate. It will also launch a dropdown menu to configure the type of search that it wants to be set.

\newpage
\addfigure[scale=0.45]{./images/search-system}{Search class diagram}{fig:SearchClassDiagram}

\newpage
\subsection{Theatrics}\label{sec:Theatrics}

With the design in the figure \ref{fig:SearchClassDiagram}, we have almost everything ready. The main remaining problem here, is that we do not have a way to animate the pop up. GTK+ do not provide any way for doing such animations, the only way is to create it by using a timeout and make the animation in each iteration. In the figure \ref{fig:Theatrics}, we can see the design for a custom framework to help to achieve this it.

\addfigure[scale=0.40]{./images/theatrics}{Theatrics}{fig:Theatrics}

\subsubsection{GeditTheatricsActor}

Wraps a specific object to animate it. Below are shown the methods of this class.

\begin{table}[H]
  \begin{center}
    \begin{tabularx}{\textwidth}{|X|X|}
      \firsthline
      \textbf{Method name:} & \textbf{Explanation:} \\
      \hline
      reset & Resets the actor and sets the new duration. \\
      \hline
      get\_target & Gets the specific object that it is animating. \\
      \hline
      get\_expired & Gets if the animation has expired. \\
      \hline
      get\_can\_expire & Gets if the animation can expire. \\
      \hline
      set\_can\_expire & Sets if the animation can expire, if it can not expire the animation behaves like a loop. \\
      \hline
      get\_duration & Gets the duration of the animation. \\
      \hline
      get\_start\_time & Gets the initial time of the animation. \\
      \hline
      get\_frames & Gets the number of frames. \\
      \hline
      get\_frames\_per\_second & Gets the number of frame in a second. \\
      \hline
      get\_percent & Gets the percent of the already done animation. \\
      \lasthline
    \end{tabularx}
    \caption{Actor methods explanation}
  \end{center}
\end{table}

\newpage
\subsubsection{GeditTheatricsStage}

It is the class that manages the actors. This class stores them, creates the timeouts for the animations and emits a signal \emph{actor-step} indicating the steps of the animation from each actor.

\begin{table}[H]
  \begin{center}
    \begin{tabularx}{\textwidth}{|X|X|}
      \firsthline
      \textbf{Method name:} & \textbf{Explanation:} \\
      \hline
      add & Adds an object to be animated and it wraps it into an actor. \\
      \hline
      add\_with\_duration & Utility method that makes the same as add but also establishing a specific duration for the actor. \\
      \hline
      remove & Removes the object from the stage to not animate it anymore. \\
      \hline
      set\_playing & Sets whether the animation is running or not. \\
      \hline
      get\_playing & Gets whether the animation is running or not. \\
      \lasthline
    \end{tabularx}
    \caption{Stage methods explanation}
  \end{center}
\end{table}

\section{Implementation}

To implement this feature, a new branch was used to make it easy to develop it and test. When it was ready it was squashed into a single patch and attached to the bug report below.

\subsection{Bug}

\url{http://bugzilla.gnome.org/show_bug.cgi?id=419805}

\newpage
\section{Tests}

\begin{table}[H]
  \begin{center}
    \begin{tabularx}{\textwidth}{|X|X|l|}
      \firsthline
      \textbf{Test:} & \textbf{Expected result:} & \textbf{Test passed?} \\
      \hline
      Press \emph{Control + F} or \emph{Search $\to$ Find} & The search pop up appears, sliding out with an animation & Yes \\
      \hline
      Write some text that appears in the document & The first occurrence is selected and the others are highlighted & Yes \\
      \hline
      Write some text that do NOT appear in the document & The search entry color changes to an error color and no text is selected or highlighted & Yes \\
      \hline
      With several occurrences highlighted, press the Up button, or the Up arrow key from the keyboard & The previous occurrence should be selected & Yes \\
      \hline
      With several occurrences highlighted, press the Down button, or the Down arrow key from the keyboard & The next occurrence should be selected & Yes \\
      \hline
      Pressing \emph{ESC} with the search pop up opened & The search dialog should slide out and go back to the position before it was opened & Yes \\
      \hline
      Pressing \emph{Enter} with the search pop up opened & The search dialog should slide out and keep the cursor in the selected occurrence & Yes \\
      \hline
      Configure the dropdown menu & The changes are set correctly and the text is searched in relation to those changes & Yes \\
      \lasthline
    \end{tabularx}
    \caption{Tests for new search system}
  \end{center}
\end{table}
