% mainfile:main.tex

\chapter{New search system}

\section{Analysis and Design}

\addfigure[scale=0.75]{./images/gedit-old-find}{Old search dialog}{fig:OldFind}

As we can see in the figure \ref{fig:OldFind}, gedit's search dialog has not evolved over the past years. This way for presenting to the user a dialog for searching has a really big problem, it obscures the text you are searching for. In the past few years a few applications like Firefox or Chrome have come with with some ideas, which are the ones used right now by most software applications for searching in the documents.

In the figure \ref{fig:FirefoxSearch}, it is shown the method Firefox uses for prompting the search to the user. This method has a few problems.
\begin{itemize}
  \item Some studies showed that presenting visual elements to the user in the bottom of a screen are less discoverable than elements in the top-left of the screen.
  \item It resizes the web browser viewer to present the search widget. This also has an advantage, it will not obscure any part of the document.
  \item It takes a big part of horizontal space.
\end{itemize}

\addfigure[scale=0.5]{./images/firefox-search}{Firefox search}{fig:FirefoxSearch}

As shown in the figure \ref{fig:ChromeSearch}, Chrome comes with an idea based on the Firefox search, but trying to fix some of the problems introduced by the previous one. Although this ones is not perfect either.
\begin{itemize}
  \item It shows the pop up in the top-right, which is not the top-left as the studies demonstrated. Although, this is still better than in the bottom of the screen.
  \item It obscures part of the document.
\end{itemize}

\addfigure[scale=0.5]{./images/chrome-search}{Chrome search}{fig:ChromeSearch}

In the analysis of the currently used search systems, we can see that there is no perfect system to manage a search by the user. In gedit, after some discussion, we decided that the best option for us would be a Chrome like search widget, as at the end it is not like people are writing that much in a text editor horizontally but doing it vertically. This means that poping out a widget on the top-right of the screen would not be that bad for the user. Apart from this problem, it was the less problematic search way for our understanding.



\addfigure[width=\textwidth]{./images/search-system-old}{Old class diagram}{fig:OldClassDiagram}

\addfigure[scale=0.45]{./images/search-system}{Search class diagram}{fig:SearchClassDiagram}

\addfigure[scale=0.5]{./images/theatrics}{Theatrics}{fig:Theatrics}

\section{Implementation}

\subsection{Bug}

\url{http://bugzilla.gnome.org/show_bug.cgi?id=419805}

\section{Tests}

\begin{table}[H]
  \begin{center}
    \begin{tabularx}{\textwidth}{|X|X|l|}
      \firsthline
      \textbf{Test:} & \textbf{Expected result:} & \textbf{Test passed?} \\
      \hline
      Press \emph{Control + F} or \emph{Search $\to$ Find} & The search pop up appears, sliding out with an animation & Yes \\
      \hline
      Write some text that appears in the document & The first occurrence is selected and the others are highlighted & Yes \\
      \hline
      Write some text that do NOT appear in the document & The search entry color changes to an error color and no text is selected or highlighted & Yes \\
      \hline
      With several occurrences highlighted, press the Up button, or the Up arrow key from the keyboard & The previous occurrence should be selected & Yes \\
      \hline
      With several occurrences highlighted, press the Down button, or the Down arrow key from the keyboard & The next occurrence should be selected & Yes \\
      \hline
      Pressing \emph{ESC} with the search pop up opened & The search dialog should slide out and go back to the position before it was opened & Yes \\
      \hline
      Pressing \emph{Enter} with the search pop up opened & The search dialog should slide out and keep the cursor in the selected occurrence & Yes \\
      %FIXME: missing the dropdown
      \lasthline
    \end{tabularx}
    \caption{Tests for new search system}
  \end{center}
\end{table}
