% mainfile:main.tex
% revised by dnolivieri
\chapter{Technical description}

% FIXME: draft

\section{Methodology}\label{sec:Methodology}

When joining an existing project, already managed by a team, it is necessary to become accustomed to the work dynamics of that 
team. This is particularly true of open source project, such as gedit, where all the maintainers (developers) work 
volantarily.  Discovering a software design methodology that fit the estabilished requirements of the team, 
was complicated.  As such, the one that best mapped the group development was  \emph{Iterative and incremental development}. 
This is because the project has dates, set by \GNOME, that we worked incrementally to achieve, through 
small changes, refactoring and redesigning when necessary.  

\addfigure{./images/iterative-development}{Iterative and incremental development diagram}{fig:IterativeDevelopment}

%TODO review this paragraph

In Figure \ref{fig:IterativeDevelopment}, the different processes of an iterative and incremental 
development are shown.  \GNOME schedules six months for each development cycle. This estabalishes the initial planning. 
Subsequently, for each feature, the team establishes a more detailed planning, acquires requirements if not already set, 
and undergoes a brainstorming session \footnote{Brainstorming is a group creativity technique, by which a group tries to find a 
solution for a specific problem by gathering a list of ideas spontaneously contributed by its members.} with the 
rest of the team or taken into account from a specific bug (see \ref{sec:Bugs}), design, testing and evaluation. 
Each feature will be placed (pushed) on the public repository,  so that other people can test it and give their opinions. 
If needed, a new release could be made(see the appendix \ref{chap:Release}).  For this, \GNOME has a specific calendar indicating 
which days each application must make a new release, so that linux distributions (i.e Ubuntu or fedora) 
know ahead of time when to create new packages for it.

The main problem with the iterative development methodology is that it lacks software engineering tools 
to represent the design and analysis.  Such tools are often needed, so some features from UML have been used as well.

\section{Programming languages}\label{sec:ProgrammingLanguages}

Below there is a list of programming languages and tools that have been used in this project:
\begin{itemize}
  \item \emph{C}: C is the recommended programming language by the platform. Using GObject we can provide to it Object 
Oriented features. This is the programming language used in the core of gedit and it will be the main one to be used in this project.
  \item \emph{Python}: The fact that we are using such a low level programming language for the core, pushed the 
developers to provide other easier to use one for plugin writers. It makes the plugins easier to maintain and makes it very fast to code.
  \item \emph{Autotools}: The fact of using autotools for the project building, means that an extensive knowledge of 
Makefiles and configure scripts is needed.
  \item \emph{Xml}: For some user interfaces, like dialogs this markup language is needed.
\end{itemize}
