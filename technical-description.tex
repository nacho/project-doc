% mainfile:main.tex
% revised by dnolivieri
\chapter{Technical description}


Despite the apparent simplicity of the \emph{gedit} application for the end user, it 
is actually a highly complex software system, having its special idiosyncratic 
architectural solutions.   Before describing the actual work of this project, a brief 
description of the \emph{gedit} architecture shall be discussed, so as to better appreciate 
the work involved in the upgrade.


\section{Gedit Architecture}

The \emph{gedit} application is written in the \emph{C} language and built around the 
GObject paradigm, invented by \GNOME, and thereby endowing the language with object-oriented constructs. 
Figure \ref{fig:geditArch} shows a simple architectural overview of the \emph{gedit} internals.

\addfigure[width=0.60\textwidth]{./images/old-architecture}{gedit architecture}{fig:geditArch}

\paragraph{GConf}

The facto standard for GObject applications to store its settings in a database 
providing applications with a signal system to detect whether a specific setting 
has been modified externally, allowing to the application to adapt itself to the 
change. This helped \GNOME applications to be simple in relation to the user interface, 
not providing too many options, but giving the possibility to change advanced options 
externally from the GConf database with its utility \emph{gconf-editor}.
GConf has become deprecated and one major change for the Project is to remove 
this dependency on the architecture, as right now all the settings are intrinsic 
in all parts of \emph{gedit}.

\paragraph{GLib}

GLib provides data structure handling for C, portability wrappers, and interfaces 
for such runtime functionality as an event loop, threads, dynamic loading, and an object 
system. It is the main reason gedit is written in \emph{C} as it has been the 
promoted language by the platform and ten years ago the only supported language 
by \GNOME. See section \ref{sec:GLib} for more information about this library.

\paragraph{Cairo}

Cairo is a 2D graphics library designed to produce consistent output on all output 
media while taking advantage of display hardware acceleration when available. It 
has a direct used by the GTK+ library and applications like gedit using GTK+.

\paragraph{GTK+}

Multi-platform toolkit used in \emph{gedit} which allows to create graphical interfaces.
The architecture is totally tied up to this library, as gedit makes direct use of its 
API in the core. The fact that we depend directly on its API it means that gedit 
needs a lot of work in order to work with the new features and changes provided 
on this library and in some cases modify and fix the library to make it work correctly 
for gedit. See section \ref{sec:GTK} for more information about GTK+.

\paragraph{GtkSourceView}

GtkSourceView is a library maintained by the \emph{gedit team} which extends the main 
multiline text editing widget. It features syntax highlighting, completion framework, 
line marks... In general what a source editor needs. Gedit makes use of this library 
and it is tied to it architecturally and in the design. Part of the project gets some 
focus on this library.

\paragraph{gedit}

This is the core. Where the project take more focus and where most of the modification 
are made. Parts of the core are explained in the Project in farther sections of this 
document.

\paragraph{Plugins}

\emph{gedit} emphasizes simplicity but it can be also extended in a powerful way 
by plugins. Figure \ref{fig:pluginsArch} shows in a really simplistic way how plugins 
are tied up in an architectural way.

\addfigure[width=0.60\textwidth]{./images/old-plugins-architecture}{gedit plugins architecture}{fig:pluginsArch}

We can see that they make direct use of static python bindings, being very attached 
to them. With the GObject-introspection work this layer is removed, making use of a 
slightly different API provided by the dynamic bindings. This project shows how it has 
been done and the amount of work required.


\section{Programming languages used in Project}
\label{sec:ProgrammingLanguages}

Before describing the exact technical contribution of the project, it is useful to briefly mention 
the programming languages and tools that were needed.  

\paragraph{C}

The \emph{C} programming language is used in the core of \emph{gedit}.   As such, it is 
the recommended language of choice for the platform.  By using GObjects, several 
object oriented features are exposed to the developer.  Apart from the technical advantages, 
these features greatly aide the organization as well as readablility of the code.
All upgrades that were carried out were written in the \emph{C} language with explicit
use of the GObjects paradigm. 

\paragraph{Python}

Python is a powerful object oriented interpreted language that provides powerful 
C- level bindings, and is not only useful as a stand-alone language but can be used 
as a plugin - development language, for its ease of use.   By wrapping the core 
C API, plugin development in python is fast, easy to develop and maintain, and 
can avoids the heavy learning curve associated with the low-level C programming 
associated with \emph{gedit}.

Python has been used in this project mainly to develop and update the powerful 
plugins provided by \emph{gedit}. It has been also used to provide the specific 
unit tests and overrides into pygobject in order to get new functionalities needed by 
\emph{gedit} plugins. The overrides are methods or classes that from a direct 
bind from C to Python do not result as convenient as they should and by adding 
overrides the powerful of Python can be used in advantage by for example converting
some specific classes in iterated classes.


\paragraph{Tools and Formats}


An important part of the development for an open-source project such as \emph{gedit} is 
the use of complex build tools.   For this, Autotools is used. 
Together with the Makefile utilities, the Autotools suite are used for creating complex
compile dependencies and can be used for making code portable across several different 
types of platforms. For a deeper explanation of Autotools, refer to section \ref{sec:autotools}.

As with many modern software applications, XML markup language is used for 
rapid portability as well as for data persistency. XML is used to 
define the user interface of some parts of \emph{gedit} making it easier to
modify without the need to touch any source code and without the need to 
rebuild anything making it faster to develop.


\section{Methodology}\label{sec:Methodology}

The methodology followed by this project has been largely dictated by that compatible 
with open Source software development.    As such, the one that best mapped the group 
development was  \emph{Iterative and incremental development}.
The typical workflow of a  \GNOME project includes incrementally small changes 
as well as refactoring and redesigning when necessary.

When joining an existing project, already managed by a team, it is necessary to 
become accustomed to the work dynamics of that team. This is particularly true of 
open source project, such as gedit, where all the maintainers (developers) work 
voluntarily.  Discovering a software design methodology that fit the established 
requirements of the team, was complicated.

\addfigure[width=0.70\textwidth]{./images/iterative-development}{Iterative and incremental development diagram}{fig:IterativeDevelopment}

%TODO review this paragraph

In Figure \ref{fig:IterativeDevelopment}, the different processes of an iterative and 
incremental development are shown.  \GNOME schedules six months for each development 
cycle. This establishes the initial planning. Subsequently, for each feature, the 
team establishes a more detailed planning, acquires requirements if not already set, 
and undergoes a brainstorming session \footnote{Brainstorming is a group creativity 
technique, by which a group tries to find a solution for a specific problem by gathering 
a list of ideas spontaneously contributed by its members.} with the rest of the team 
or taken into account from a specific bug (see \ref{sec:Bugs}), design, testing and 
evaluation. Each feature will be placed (pushed) on the public repository,  so that 
other people can test it and give their opinions. If needed, a new release could be 
made(see the appendix \ref{chap:Release}).  For this, \GNOME has a specific calendar 
indicating which days each application must make a new release, so that Linux 
distributions (i.e Ubuntu or Fedora) know ahead of time when to create new packages for it.

The main problem with the iterative development methodology is that it lacks software 
engineering tools to represent the design and analysis.  Such tools are often needed, 
so some features from UML have been used as well.


\section{Details of the Project}

In this project several objectives have been tried to achieve, and now that we have 
an overview of the current architecture of \emph{gedit}, we can get a more detailed 
vision of the developed code for each of the objectives. Refer to the chapter 
\ref{chap:Objectives} for more information.

\paragraph{Remove Old API}

As seen in the architecture figure \ref{fig:geditArch}, gedit plugins depend 
on the stability of the gedit API. Removing public API in a project that it is 
exposing it externally may deal to broken plugins that will may not run in future 
versions of \emph{gedit}. This task apart from removing all the deprecated API 
meant to update the plugins shipped by gedit and \emph{maintained} by the gedit team.

\paragraph{New highlighting language files}

The fact that gedit is a text editor also used to visualized source code, that
GtkSourceView is intrinsic in gedit's architecture and that GtkSourceView is 
maintained by gedit's team, it means that it is necessary to update it and 
create new highlighting language files to make the edition for the user easier 
and also to attract new users that may want to develop a specific feature in 
that specific programming language.

GtkSourceView provides an easy way to create new highlighting language files 
using the XML format to define them. The main problem here is that in order to 
create or to update a language file it needs a lot of \emph{research} and good 
\emph{understanding} of the language file that you want to create or update.

\paragraph{Tab Groups}

Tab groups is a new feature implemented on this Project which allows users to
visualize several documents side by side without the need of having several
gedit windows. This has been implemented in the gedit core and the main problem
here was to implement it a way so there would not be any need to change the plugins
in order to allow this feature.
That is the main reason it was implemented as a proxy between the single tab groups
to the gedit's main window. Without the need to expose any external API to deal
with it. For more information about this topic please refer to chapter \ref{chap:TabGroups}.

\paragraph{Manage invalid characters}

One of the main problems of depending on GTK+ is that it does not support to 
visualize invalid UTF-8 characters in the multiline text editing widget that provides
by default. As it is one of the most requested features, to visualize even the 
text files which contain not valid characters, gedit has to deal with and show 
them to the user in some characteristic way so the user can detect which ones 
are not correct. This task is quite hard as it is a work at a very low level 
and it touches very difficult parts of the core of gedit. For more information 
how this was implemented in this project refer to the chapter \ref{chap:InvalidChars}.

\paragraph{New drag and drop system}

\emph{gedit} is a very old application and has been evolving with GTK+ and GTK+
with it. The custom drag and drop system used by gedit till now it was implemented 
in Epiphany\footnote{Epiphany is the default web browser of the \GNOME desktop} 
and then taken by gedit years ago. GTK+ sometime ago implemented its own drag 
and drop system but it lacked some of the features needed by gedit. After requesting 
this features and get them inside GTK+ it has been decided to drop the custom 
drag and drop system and use the one in GTK+ which apart from being more robust, 
it also provides a better integration with the desktop. See chapter \ref{chap:DND}
for more information on how this was developed in this project.

\paragraph{New search system}

The search system of gedit has been one of the relics from the old times when 
everything had to be dialog. For several years several times has been tried to 
implement a new search system for it without lucky to get it right. With the 
improvements in GTK+ and the new ideas provided by applications like Chrome or 
Firefox it was possible to create a new search system, which as explained in the 
chapter \ref{chap:SearchSystem}, is not perfect but it is more handy than the
previous way provided by gedit to search on documents. This new search system 
had the main problem or requiring an animation to visualize and to dismiss which 
required a lot of thinking to implement correctly and functional, having the fact 
that GTK+ does not provide any way to manage animations easily.

\paragraph{GTK+ 3}

As explained before in the architecture, GTK+ is the base of gedit, it provides 
all the widgets used for the user interface. If GTK+ evolves, gedit as part of 
the \GNOME desktop must evolve with it and help it to be better. This project 
shows some parts where it was helped to the GTK+ library and some of the main 
changes on it which made so complicated at some point to depend on GTK+ 3 and be 
up to date making gedit build. See chapter \ref{chap:GTK3} for more information.

\paragraph{GSettings}

In the architecture section in this chapter, it can be seen that GConf is one 
of the intrinsic parts of gedit's architecture. With the introduction of GSettings 
in this cycle, GConf has become deprecated, which means changing one of the parts 
in the core and in the plugins used everywhere. Removing a part of the architecture 
and changing it with another it is always a hard task. It needs to know pretty well 
what you are changing as it could deal to break parts of the system that you are not 
counting with. See chapter \ref{chap:GSettings} for more information on how this 
task has been made.
