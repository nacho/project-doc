% mainfile:main.tex
% revised by dnolivieri
\chapter{Technical description}


Despite the apparent simplicity of the \emph{gedit} application for the end user, it 
is actually a highly complex software system, having its special idiosyncratic 
architectural solutions.   Before describing the actual work of this project, a brief 
description of the \emph{gedit} architecture shall be discussed, so as to better appreciate 
the work involved in the upgrade.  


\section{Gedit Architecture}

The \emph{gedit} application is written in the \emph{C} language and built around the 
GObject paradigm, invented by \GNOME, and thereby endowing the language with object-oriented constructs. 
Figure \ref{fig:geditArch} shows a simple architectural overview of the \emph{gedit} internals. 

% FIXME
%\begin{figure}
%\epsfig{geditArch}
%\caption{The \emph{gedit} architecture.}
%\label{fig:geditArch}
%\end{figure}


**** This section should answer the following questions: 

   - what are the essential elements of the gedit architecture?
   - how does the language influence the architecture (if at all)?
   - introduce/identify (as if foreshadowing) the elements of the architecture that 
     you have you needed to use directly ?   Here you are trying to prepare the 
     reader for what you will describe as your direct contribution....




\section{Programming languages used in Project}
\label{sec:ProgrammingLanguages}

Before describing the exact technical contribution of the project, it is useful to briefly mention 
the programming languages and tools that were needed.  

\paragraph{C} 

The \emph{C} programming language is used in the core of \emph{gedit}.   As such, it is 
the recommended laguage of choice for the platform.  By using GObjects, several 
object oriented features are exposed to the developer.  Apart from the technical advantages, 
these features greatly aide the organization as well as readablility of the code.
All upgrades that were carried out were written in the \emph{C} language with explicit
use of the GObjects paradigm. 

\paragraph{Python}

Python is a powerful object oriented interpreted language that provides powerful 
C- level bindings, and is not only useful as a stand-alone language but can be used 
as a plugin- development language, for its ease of use.   By wrapping the core 
C API, plugin development in python is fast, easy to develop and maintain, and 
can avoids the heavy learning curve associated with the low-level C programming 
associated with \emph{gedit}.


**** mention briefly how you used python... for example if you used it 
  in the development of plugins, or if you actually developed some 
  of the plugin interface.???




\paragraph{Tools and Formats}


An important part of the development for an open-source project such as \emph{gedit} is 
the use of complex build tools.   For this, Autotools is used.   
Together with the Makefile utilities, the Autotools suite are used for creating complex
compile dependencies and can be used for making code portable across several different 
types of platforms.  

As with many modern software applications, XML markup language is used for 
rapid portability as well as for data persistency.  ....



***   Mention other ways which you used this ... 











\section{Methodology}\label{sec:Methodology}

The methodology followed by this project has been largely dictated by that compatible 
with open Source software development.    As such, the one that best mapped the group 
development was  \emph{Iterative and incremental development}. 
The typical workflow of a  \GNOME project includes incrementally small changes 
as well as refactoring and redesigning when necessary.  



When joining an existing project, already managed by a team, it is necessary to become accustomed 
to the work dynamics of that team. This is particularly true of open source project, such as gedit, 
where all the maintainers (developers) work voluntarily.  Discovering a software design methodology 
that fit the established requirements of the team, was complicated.  




\addfigure{./images/iterative-development}{Iterative and incremental development diagram}{fig:IterativeDevelopment}

%TODO review this paragraph

In Figure \ref{fig:IterativeDevelopment}, the different processes of an iterative and incremental 
development are shown.  \GNOME schedules six months for each development cycle. This establishes the initial planning. 
Subsequently, for each feature, the team establishes a more detailed planning, acquires requirements if not already set, 
and undergoes a brainstorming session \footnote{Brainstorming is a group creativity technique, by which a group tries to find a 
solution for a specific problem by gathering a list of ideas spontaneously contributed by its members.} with the 
rest of the team or taken into account from a specific bug (see \ref{sec:Bugs}), design, testing and evaluation. 
Each feature will be placed (pushed) on the public repository,  so that other people can test it and give their opinions. 
If needed, a new release could be made(see the appendix \ref{chap:Release}).  For this, \GNOME has a specific calendar indicating 
which days each application must make a new release, so that Linux distributions (i.e Ubuntu or fedora) 
know ahead of time when to create new packages for it.

The main problem with the iterative development methodology is that it lacks software engineering tools 
to represent the design and analysis.  Such tools are often needed, so some features from UML have been used as well.





***************\section{Details of the Project}


Nacho:  Here you have to put the details of the code that you developed....


I suggest that you take the objective section one by one and write out 
a technical description for each of the goals... making connection with the 
section on architecture....


***************************






