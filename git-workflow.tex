% mainfile:main.tex
%  revised by dnolivieri
\chapter{Work flow using Git}

\emph{Git} is a very powerful tool, used by many important projects, such as the \emph{Linux kernel}, and has been chosen for \GNOME desktop. The fact that it is so powerful does not mean that it is very easy to use. Here the main steps shall be described in order to have an overview of the main work flow.

Before explaining the main steps, it must be known that git manages everything with branches. The main branch, usually used for the development copy of the program is called: \emph{master}.

\section*{Get a copy of the software}

To do this, the  `clone' command is used 


\noindent\graybox{git clone git://git.gnome.org/gedit}

\section*{Committing a change}

In git, this operation is slightly different from svn, and is split into two steps:

\noindent\graybox{git commit -a}

This command will make the change in the local branch, creating the specific commit. Note, that that we also specify the `-a' argument so all changes in the branch are committed. If we want to stage specific changes we can use the command:

\noindent\graybox{git add file}

\noindent\graybox{git push origin branch\_name}

This other command will put the previous commit or commits in the remote branch.

\section*{Update the working copy}

\noindent\graybox{git pull --rebase}

Note that in this case it is important to pass `--rebase', since for any commit in this specific branch before updating, 
you will want that commit to be in the top of the updated branch.


\section*{Wipe out previous changes}

\noindent\graybox{git checkout -f}

This command will remove all the changes made in the current branch.

\noindent\graybox{git reset}

The previous command removes the changes made in a branch, although git reset allows the user to drop out the specific commits you don't want.

\section*{Create a new branch}

\noindent\graybox{git checkout -b branch\_name}

It will create a new branch with the name `branch\_name'.

\section*{Switch between branches}

\noindent\graybox{git checkout branch\_name}

Note that to be able to change the branch, no changes on it are required.

\newpage
\section*{Merge branches}

\noindent\graybox{git merge branch\_name}

With this command, a direct merge will be made, placing the commits in the places where they were made, and if there is any conflict a new commit will be made on the top fixing this conflicts.

\noindent\graybox{git rebase -i branch\_name}

It puts all the commits to be merged on top of the branch desired.  If there are conflicts, each commit with a conflict should be edited and fixed.

\section*{Remove a branch}

\noindent\graybox{git branch -D branch\_name}

This one will remove the local branch named with branch\_name.

\noindent\graybox{git push origin :branch\_name}

And this will remove branch\_name in the remote server.

\section*{Creating a patch}

\noindent\graybox{git diff \textgreater~patch\_name}

This command creates a file with patch\_name as name, with the local non committed changes.

\noindent\graybox{git format-patch origin}

This command will create a patch for each new commit not pushed to the remote branch. The difference with the previous command is that this one conserve extra information, such as the author of the commit.

\newpage
\section*{Apply a patch}

\noindent\graybox{git apply patch\_name}

This command applies the patch on top of the local branch.

\noindent\graybox{git am patch\_name}

This command applies the patch but it also creates the local commit with the extra information.
