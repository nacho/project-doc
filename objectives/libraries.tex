% mainfile:../main.tex

\section{Port to new libraries}\label{sec:NewLibraries}

\subsection{GTK+ 3}\label{sec:GTK3}

A significant change in the gedit application, provided by this project, has been the migration from GTK+ 2 to GTK+ 3 
(for a more detailed description of GTK+, please refer to section \ref{sec:GTK}).
This task required the elimination of all deprecated software code with dependence upon GKT+2, updating of gedit to the 
new APIs, and most importantly, follow the evolution of GTK+ 3, which has been a moving target.  Thus, if an error was 
detected in GTK+, then it was necessary to submit a bug report for it and possibly provide a patch to fix it (Please 
refer to \ref{sec:Patches} for an explanation on how to do this correctly).  A direct result of using the a library 
that \textit{itself} is under a fierce development phase, is that continual testing was needed during all stages of 
development and that at no point could library functionality be taken for granted.


\subsection[GSettings]{GSettings\cite{website:gio}}\label{sec:GSettings}

Amongst the specific tasks associated with the porting of gedit to GTK+ 3, it was necessary 
to to use GSettings, instead of the previously used GConf.  GSettings is part of glib (see \ref{sec:GLib}) 
and provides a convenient API for storing and retrieving application settings.   This 
change is motivated by the following reasons: 
\begin{itemize}
  \item GConf has been deprecated.
  \item GSettings provides better API to be used by GObject-based applications.
  \item It is used with D-Bus\footnote{D-Bus is a message bus system, a simple way for applications to talk to one another.}.
  \item The storage system is implemented by different backends, in relation to the operative system or the choose of the user.
\end{itemize}


\subsection{libpeas}\label{sec:libpeas}

Another specific task that has been accomplished in this project is related to a removal of the 
custom plugin system in favor of the \texttt{libpeas} library.  A more detailed description of the 
libpeas library can be found in section \ref{sec:peas}, as well as a motivation for its substitution 
over the previous custom library.


\subsection{GObject-introspection}\label{sec:GObjectIntrospection}

Finally, another major change provided by the work of this project associated with libraries 
is the use of GObject-introspection, which retrieves metadata information from the public API of 
libraries and makes it doable to create dynamic bindings for it. Such that you write a class in
C and without any additional work you can make use of that class inside Python.

A further discussion of GObject-introspection is provided in section \ref{sec:g-i}.

