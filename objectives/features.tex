% mainfile:../main.tex

\section{New features}\label{sec:NewFeatures}

Besides the major library changes provided by this project, that are hidden from the 
user, there are several more visible changes undertaken in this project, that have been 
realized.  Each of the major feature changes is described below.


\subsection{Remove old API}\label{sec:RemoveOldAPI}

For several years, developers of \emph{gedit} have opted for maintaining a stable unchanging 
API interface for plugin - developers. The implication of this cautious design choice in the 
evolution of \emph{gedit} has been a continual growth in deprecated and duplicated code 
in order to lessen the burden of plugin developers need to update to the new API features. 
Thus, if some part of the API was deprecated, it would be marked as such, yet maintained 
in the source tree, still available to third party plugins, without the need for update action.
A result of such a policy is .....

The objectives of this project for providing a major release of \emph{gedit}, removes 
all references to the old API.  Thus, several older plugins will cease to be supported 
unless updated.


\subsection{New highlighting language files}\label{sec:HighFiles}

A new feature, provided by software written for this project, is the use of 
highlighting language files.   Since it is common to receive requests from 
users for highlighting for new programming files, (....why?? )

Thus, this project has added the use of this feature, to fix, review and 
add new language files to highlight syntax for new programming languages.


\subsection{Tab Groups}\label{sec:TabGroups}

An important editor feature, that is clearly desirable in a full feature application 
such as \emph{gedit}, is to be able to edit multiple documents side-by-side. 
Such a feature is useful for comparison, and is common in applications such as emacs. 
Thus, this feature has been added as a design objective for this project.


\subsection{Manage invalid characters}\label{sec:InvalidChars}

GTK+ is UTF-8\footnote{UTF-8 is a multibyte character encoding for Unicode.} oriented. There 
are several characters\footnote{An invalid byte sequence is a sequence of bytes that can not be 
together in an UTF-8 encoded system.} in UTF-8 that cannot be introduced in GTK+ widgets. 
In \emph{gedit} the documents that contain invalid characters are treated as binary files and not 
opened, this is sometimes not good when you have a wrongly encoded document and you want to fix it. 
This feature will allow to open this kind of documents.

\subsection{Use new Drag and Drop system}\label{sec:DND}

The present stable version of \emph{gedit} uses a custom implementation to make 
drag and drop of opened tabs.  Given that the new GTK+ supports all the functionalities 
needed by gedit, the custom drag/drop code is no longer required and should take advantage
of the more robust GTK+ API.  Moreover, the GTK+ API allows for better integration 
with the desktop and provides for a cleaner implementation of the \emph{gedit} source code.
As such, this is a major feature contribution provided by this project. 

\subsection{New search system}\label{sec:SearchSystem}

The present method that \emph{gedit} searches files is obsolete and does not take advantage of 
the latest algorithms nor graphical user interface philosophies, such as embedding the search 
entry in the bottom of the window (as Firefox) or sliding out the search entry (as with Chrome). 
Instead, the \emph{gedit} has used a dialog for searching, with all its associated problems. 

An important contribution of this project has been to change the manner of searching text 
within documents.
