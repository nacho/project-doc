% mainfile:main.tex
% revised by dnolivieri
\chapter{Tab groups}

One of the main features required for any \GNOME application,  is to provide an easy to use user interface. In order to manage the tab groups, 
three menu items were provided, as shown in Figure \ref{fig:DocumentsMenu}. The following is an explanation of each menu item:
\begin{itemize}
  \item \textbf{New Tab Group:} It creates a new tab group in the right side of the active document, with a new document on it.
  \item \textbf{Previous Tab Group:} Moves the focus to the tab group on the left. If there is no tab group on the left it moves the focus to the last tab group.
  \item \textbf{Next Tab Group:} Moves the focus to the tab group on the right. If there is no tab group on the right it moves the focus to the first tab group.
\end{itemize}

\addfigure[scale=0.50]{./images/documents-menu}{Documents menu}{fig:DocumentsMenu}

In Figure \ref{fig:UserTabGroups}, an example of the appearance of a gedit window with two tab groups and several docuements opened on each of them is shown.

\addfigure{./images/user-tag-groups}{Tag groups}{fig:UserTabGroups}
