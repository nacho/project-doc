% mainfile:main.tex

\chapter{Resumen en Español}

\section{Introducción}

Desde el inicio del proyecto \GNOME, \emph{gedit} ha sido siempre el editor
de textos por defecto de \GNOME. Mientras que el escritorio ha ido evolucionando
y madurando, \emph{gedit} no ha avanzado al mismo nivel. Por eso, este proyecto
pretende remediar esta situación añadiendo nuevas características y actualizando
la aplicación para utilizar las nuevas librerías propuestas para el desarrollo
de la nueva versión de \GNOME.

\section{Objetivos}

El objetivo de este proyecto es el incrementar la funcionalidad de \emph{gedit},
añadiendo nuevas características y portando estos cambios a las nuevas librerías
marcadas como un requisito para formar parte de \GNOME 3.0.

\subsection{Nuevas librerías}

\subsubsection{GTK+ 3}

Un cambio importante en la aplicación, proporcionado por este proyecto, ha sido
la migración de GTK+ 2 a GTK+ 3. Esta tarea requirió la eliminación de todo el
código fuente obsoleto y actualizar gedit a las nuevas APIs, pero lo más importante
has sido el seguir la evolución de GTK+ 3 que ha sido un objetivo muy cambiante.

\subsubsection{GSettings}

Además de tener que portar gedit a GTK+ 3, ha sido necesario también tener que
portarlo a GSettings, en lugar de la tecnología usada anteriormente GConf.
GSettings es parte de GLib y proporciona un API conveniente para almacenar y
obtener la configuración de una aplicación.

\subsubsection{libpeas}

Otra tarea específica que se ha realizado en este proyecto está relacionada
con la eliminación del sistema de complementos personalizado en favor de la librería
\texttt{libpeas}. Una descripción más detallada de esta librería se puede
encontrar en la sección \ref{sec:peas}, así como la motivación por lo cual
se sustituyó el sistema personalizado anterior.

\subsubsection{GObject-introspection}

Finalmente, otro cambio importante que se ha realizado en este proyecto y
asociado con librerías es el uso de GObject-introspection, el cual obtiene
meta-datos de la API pública de las librerías y nos proporciona una forma
sencilla de crear bindings dinámicos para ellas. De la manera que escribes
una clase en C y sin trabajo adicional puedes usar esa misma clase desde Python.

\subsection{Nuevas características}

Además de los cambios tan grandes proporcionados en este proyecto, que principalmente
están escondido de cara al usuario, hay unos cuntos cambios más visibles que se
han realizado en este proyecto.

\subsubsection{Eliminación del API obsoleto}

Durante varios años los desarrolladores de \emph{gedit} han optado por mantener
un API sin cambios para los desarrolladores de complementos. Esto ha implicado
el tener APIs obsoletas y código duplicado en el núcleo de gedit para evitar
que los desarrolladores de complementos sufrieran tener que hacer cambios en
sus trabajos. La idea en este proyecto, es eliminar toda esta API obsoleta
actualizando los complementos necesarios que todavía estaban usando la API antigua
en lugar de la nueva y de esta manera tener un núcleo de la aplicación más
limpio.

\subsubsection{Nuevos archivos de resaltado de sintaxis}

Una nueva característica proporciona en este proyecto ha sido la creación de
archivos de resaltado de sintaxis. Ya que es algo normal recibir solicitudes
de usuarios para resaltar nuevos lenguajes de programación, ya que los lenguajes
de programación evolucionan o nuevos son creados. En un editor de textos para
uso general es necesario proporcionar a los usuarios este tipo de características.

\subsubsection{Grupos de solapas}

Una característica importante para un editor, que claramente es algo necesario
para una aplicación como gedit, es el poder editar varios documentos lado a lado.
Una característica como esta es útil para comparar y es común en aplicaciones
como emacs. Por ello esta característica se ha añadido como un objetivo de diseño
para este proyecto.

\subsubsection{Gestión de caracteres inválidos}

GTK+ está basado en UTF-8. En UTF-8 hay unas secuencias de bytes que no pueden
estar continuas y que no se pueden introducir en los widgets the GTK+. En \emph{gedit}
los documentos que contenían caracteres inválidos eran tratados como archivos
binarios y no se abrían, esto algunas veces no era bueno cuando por ejemplo, se
tenía un documento que estaba codificado incorrectamente y se quería arreglar.
Esta nueva característica permite abrir este tipo de documentos.

\subsubsection{Uso del nuevo sistema de arrastrar y soltar}

En la presente versión estable de \emph{gedit} se usa una implementación para
arrastrar y soltar solapas personalizada. Dado que la nueva versión de GTK+
soporta todas las funcionalidades necesarias por gedit, la implementación
personalizada ya no se necesita y se debería de tomar ventaja del API del API
robusto de GTK+. Además, el API de GTK+ permite una mejor integración con
el escritorio y proporciona una implementación más limpia del código de \emph{gedit}.

\subsubsection{Nuevo sistema de búsqueda}

El sistema de búsqueda en archivos actual de gedit está obsoleto y no toma
ventaja de los últimos algoritmos o filosofías para interfaces de usuario
gráficas, como por ejemplo empotrar el diálogo de búsqueda en la ventana
al estilo Firefox o con una animación de entrada y salida al estilo Chrome.
En su lugar, \emph{gedit} usa un diálogo para las búsquedas con todos los problemas
asociados.

Una importante contribución de este proyecto ha sido cambiar la manera con la
cual se busca en los documentos abiertos con \emph{gedit}.

\section{Descripción técnica}

\subsection{Lenguajes de programación usados en el proyecto}

\paragraph{C}

El lenguaje de programación \emph{C} se usa en el núcleo de \emph{gedit}.
Además, es el lenguaje recomendado por la plataforma. Por medio de GObjects,
se exponen varias características de la orientación a objetos.

\paragraph{Python}

Python es un lenguaje de orientación a objetos interpretado que proporciona
bindings para C con lo cual no solo se puede usar solo sino también para
crear complementos por su facilidad de uso.

Python ha sido usado en este proyecto principalmente para desarrollar y actualizar
los complementos proporcionados en \emph{gedit}. Ha sido usado también para
proporcionar pruebas unitarias y overrides en pygobject, para poder obtener
nuevas funcionalidades necesarios por los complementos de \emph{gedit}.

\subsection{Herramientas y formatos}

Una parte importante del desarrollo para un proyecto de software libre como
\emph{gedit} es usar herramientas de construcción muy complejas. Para esto,
se usó Autotools. Junto con las utilidades Makefile, la suite Autotools
es usada para crear dependencias de compilación muy complejas y puede ser
usado para crear código multiplataforma.

Como muchas otras aplicaciones modernas, el lenguaje de marcado XML es usado
para tener rápida portabilidad así como persistencia de datos. En gedit, XML
ha sido usado para definir algunas partes de la interfaz de usuario.

\newpage
\section{Metodología}

La metodología seguida en este proyecto has sido principalmente dictada 
por el desarrollo de software libre. Por esto, la que mejor encajaba en el
grupo de desarrollo fue \emph{Desarrollo Iterativo e Incremental}.
El flujo de trabajo típico de un proyecto de \GNOME incluye pequeños cambios
incrementales así como refactorizaciones y rediseños cuando es necesario.

\section{Planificación}

Para la planificación del proyecto se empleó el programa Planner, una
herramienta para el escritorio de \GNOME similar a Microsoft Project y
que permitió hacer los diagramas de Gantt necesarios para este proyecto.

\section{Proceso de desarrollo}

Por cada característica, entendiendo por característica, un error en el
programa o una nueva funcionalidad a ser añadida se realizó lo siguiente:

\begin{itemize}
  \item Análisis
    \begin{itemize}
      \item Estudio del problema.
      \item Discusión con el resto de los desarrolladores.
      \item Crear diagramas usando ingeniería inversa si ha sido necesario.
      \item Análisis de los casos de uso a no ser que no fuese necesario para una tarea específica si la funcionalidad ya estaba hecha.
    \end{itemize}
  \item Diseño
    \begin{itemize}
      \item Estudio de las tecnologías a ser usadas por la nueva funcionalidad.
      \item Estudio del diseño actual.
      \item Generar prototipos iniciales.
      \item Tormentas de ideas con el resto de desarrolladores para comprobar cual puede ser el mejor diseño.
    \end{itemize}
  \item Codificación
    \begin{itemize}
      \item Implementación usando C con GObject o Python, dependiendo del complemento o de si es una funcionalidad en el núcleo.
      \item Reportar un bug para revisar el parche.
    \end{itemize}
  \item Pruebas
    \begin{itemize}
      \item Probar las nuevas funcionalidades.
    \end{itemize}
  \item Despliegue
    \begin{itemize}
      \item Enviar los parches al repositorio de \GNOME.
    \end{itemize}
\end{itemize}

\section{Planificación}

La planificación real del proyecto no se correspondió con la planificación
inicial en el anteproyecto, la razón de esto ha sido por lo siguiente:

\begin{itemize}
  \item La fecha de lanzamiento de \GNOME fue retrasada.
  \item La realización del Erasmus por parte del autor produjo también un retraso.
\end{itemize}

En el capítulo \ref{chap:planning} se pueden comprobar los diferentes diagramas
de Gantt relacionados con la planificación inicial y el final del proyecto.


