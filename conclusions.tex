% mainfile:main.tex

\chapter{Conclusions}

\section{About the developed product}

gedit 3.0 is a good product. It was a hard development cycle, with some delays on the development, as it was known that it could not be achieved the \GNOME 3.0 target on time, but at the end a good program was released.

One thing that can be learned from this project, is that being on the edge is very hard. You have to compile all the libraries by yourself, keep them updated, update your program to the last API changes from these libraries every day if you do not want people pissed off and modify your previous design due to the fact that the library has changed.

In my opinion for such a small group of people it is not worth it working such on the edge of the development. I have to admit that doing it helped a lot to the GTK+ and GLib team to have a better code base. But it was too complicated for us. This shows that sometimes it is better to delay some features in order to have more tested and better approached programming.

\section{About the working process}

The working process is not good at all, it missing directions from the first time. The way everybody works in this product is `too free', but the point is that it works. The project has been evolving during the last 10 years, so it is remarkable that not really following an specific methodology, not having specific plans, but knowing that you have 6 months to add new features to the program and from this point you have to decide if it is worth making it or not it kinda works.

\newpage
\section{About the personal experience}

gedit is one of the most important text editor in the current times. It is the main text editor by default of the \GNOME desktop. Being part of it is an incredible personal achievement.

Also not only about the gedit, but the community around it. The fact that you are working with such a lot of experts, learning so much from them, it is a undeniable experience.
