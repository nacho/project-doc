% mainfile:main.tex
% revised dnolivieri
\chapter{Conclusions}

\section{About the developed product}

gedit 3.0 is a good product. Although it involved an arduous development cycle, with some development delays, 
a solid program was released, even though it was foreseen that the \GNOME 3.0 initial target release date was not attainable.

The contributions I have made, make gedit a more mature project, with nicer features that will provide a better user experience. One feature that I am really proud of, is the new search system. It solves the problems that the old dialog had, it is fast to use and very well designed. Being able to open invalid documents, this is a very important feature, it provides a way to open any document, no matter it has binary symbols or it is just a damaged document which can be saved parts of it. The tab groups are also a good but they will probably not be so useful, as they fit some specific use cases for advanced users but even though it is something important to have as a way to reach to more users. Remark also all the cleanup that it has been done. Having a more stable core, organized and maintainable code will provide an easier further development.

\section{About the development cycle}

Something that can be learned from this project, is that working under stress is very difficult. It requires that all the libraries are compiled, 
kept up to dated, updated to the last API changes from these libraries every day so as not to upset other developers, and 
modify previous designs due to the fact that the library has changed.

In my opinion, working under such stress with such a small group of people is not worthwhile.   I have to admit, that doing it helped a lot to the GTK+ and GLib team to have a better code base. But it was too complicated for us. This shows that sometimes it is better to delay some features in order to have more tested and better approached programming. In the appendix \ref{GNOME3AppDevs} an overview about this is pointed out.

\section{About the working process}

The working process is not good at all, it misses directions from the first time. The way everybody works in this product is `too free', but the point is that it works. The project has been evolving during the last 10 years, so it is remarkable that not really following an specific methodology, not having specific plans, but knowing that you have 6 months to add new features to the program and from this point you have to decide if it is worth making it or not, it works and it provides a good product.
