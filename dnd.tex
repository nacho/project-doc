% mainfile:main.tex

\chapter{Drag and Drop of Tabs}

%FIXME first draft

\section{Analysis and design}

As said in the previous chapter, a \emph{modification} in the drag and drop system had to be made. gedit's custom drag and drop was implemented a few years ago, when GtkNotebook did not have a way to do it by itself. This was implemented in GTK+ since the version 2.12, but it had a problem holding the port of gedit, it did not allow to drag a tab out of the window to create a new window with this tab attached to it. This changed with the introduction of the \emph{create-window} signal from GtkNotebook, which is emitted when a tab is dropped out of the main window.

To implement this the next things are needed:
\begin{itemize}
  \item Remove all the legacy code from the custom implementation.
  \item Proxy the \emph{create-window} signal from GtkNotebook to GeditMultiNotebook.
  \item Listen to this signal in \emph{gedit-window}.
  \item When the signal is emitted clone the window and return to the signal the new active notebook, so GTK+ can know to which GtkNotebook add the dragged tab.
\end{itemize}

See that with this implementation we will still have one regression, we will not be able to drop a tab in the view. From the documentation: If you want a widget to interact with a notebook through DnD (i.e.: accept dragged tabs from it) it must be set as a drop destination and accept the target "GTK\_NOTEBOOK\_TAB". The notebook will fill the selection with a GtkWidget** pointing to the child widget that corresponds to the dropped tab.

\section{Implementation}

Once the analysis and design have been done, the implementation was straightforward, it was a matter of:
\begin{itemize}
  \item \emph{setting a group name}: Notebooks with the same name will be able to exchange tabs via drag and drop. A notebook with a NULL group name will not be able to exchange tabs with any other notebook.
  \item \emph{proxy} the signal and \emph{listen} to it.
  \item \emph{add "GTK\_NOTEBOOK\_TAB"} to the list of DnD targets of the view and implement the received for it. Code shown below.
  \item extra careful had to be put when implementing this to have the signals emitted in the \emph{right order}.
\end{itemize}

\begin{lstlisting}[style=GObject]

/* This goes in the constructor of the notebook class */
#define GEDIT_NOTEBOOK_GROUP_NAME "GeditNotebookGroup"
gtk_notebook_set_group_name (GTK_NOTEBOOK (notebook),
                             GEDIT_NOTEBOOK_GROUP_NAME);

/* Snippet of code for the received part of code needed for the view */
else if (info == TARGET_TAB)
{
	GtkWidget *notebook;
	GtkWidget *new_notebook;
	GtkWidget *page;

	notebook = gtk_drag_get_source_widget (context);

	if (!GTK_IS_WIDGET (notebook))
	{
		return;
	}

	page = *(GtkWidget **) gtk_selection_data_get_data (selection_data);
	g_return_if_fail (page != NULL);

	/* We need to iterate and get the notebook of the target
	   view because we can have several notebooks per window */
	new_notebook = get_notebook_from_view (widget);

	if (notebook != new_notebook)
	{
		gedit_notebook_move_tab (GEDIT_NOTEBOOK (notebook),
			                 GEDIT_NOTEBOOK (new_notebook),
			                 GEDIT_TAB (page),
			                 0);
	}

	gtk_drag_finish (context, TRUE, TRUE, timestamp);
}

\end{lstlisting}

\subsection{Bug}

In the next link it can be checked the patches provided and the revisions by the other developers:

\noindent\url{https://bugzilla.gnome.org/show_bug.cgi?id=456959}

\newpage
\section{Tests}

\begin{table}[H]
  \begin{center}
    \begin{tabularx}{\textwidth}{|X|X|l|}
      \firsthline
      \textbf{Test:} & \textbf{Expected result:} & \textbf{Test passed?} \\
      \hline
      Reorganize tabs & The tab was kept in the desired position & Yes \\
      \hline
      Drop a tab out of the window & A new window was created, the tab was attached to it and detached from the source window & Yes \\
      \hline
      Drop a tab in other window view & The tab is detached from the source window and attached to the target window & Yes \\
      \hline
      Create one tab group and drop a tab on it & The tab is detached from the source tab group and attached to the target one & Yes \\
      \hline
      Create one tab group and drop the last tab on another tab group & The tab is detached from the source tab group, the source tab group is removed and the tab is attached in the target tab group & Yes \\
      \lasthline
    \end{tabularx}
    \caption{Tests for new Drag and Drop system}
  \end{center}
\end{table}
