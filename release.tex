% mainfile:main.tex
% revised by dnolivieri
\chapter{Making a release}\label{chap:Release}

The first thing that you have to know when you are making a release, is whether the module is using pre-release version incrementing (i.e. bumping the version number just before release) or post-release version incrementing (which is preferred for libraries). gedit uses a pre-release version incrementing. This can be usually checked in the \emph{configure.in} or \emph{configure.ac} files. After this has been checked the next steps have to be followed:

\section*{Make sure you are up to date}

\noindent\graybox{git pull --rebase}

\section*{Make sure there are not local changes}

\noindent\graybox{git status}

\section*{Increment the version}

\noindent Modify configure.ac and modify the version.

\section*{Update the README file}

\noindent Usually you have to update the version number in all the occurrences in this document.

\section*{New entry in the NEWS file}

\noindent The ChangeLog should be read a bullet point for each significant change together with credit to the person 
who did the work should be formulated. Below, an example is provided together with the command to show the ChangeLog:

\noindent\graybox{git log}

\begin{lstlisting}[style=plain]
============
gedit 2.91.6
============

New Features and Fixes
======================
- Fix encoding conversion (Ignacio Casal Quinteiro)
- Add invalid char support (Ignacio Casal Quinteiro, Paolo Borelli)
- More plugins ported to pygobject
- Misc bugfixes

New and updated translations
============================
- cs (Marek Cernocky)
- eo (Kristjan SCHMIDT)
- gl (Fran Dieguez)
- pa (A S Alam)
\end{lstlisting}

\section*{Recompile everything}

For this task, several commands have to executed:

\noindent\graybox{./autogen.sh --enable-gtk-doc}

\noindent\graybox{make}

\noindent\graybox{make install}

\newpage
\section*{Create the tarball\footnote{A tar file or compressed tar file is commonly referred to as a tarball.}}

The command below will create the tarball and automatically check that all files are added correctly to this compressed file, that everything compiles correctly and that all tests are passed.

\noindent\graybox{make distcheck}

Once distcheck finished, it should show something like this:

\begin{lstlisting}[style=plain]
==================================================
gedit-2.90.6.tar.xz is ready for distribution
==================================================
\end{lstlisting}

\section*{Commit}

\noindent\graybox{git commit -a}

\section*{Push the change}

\noindent\graybox{git push origin master}

If some change has been done in the repository before this step and the commit can not be pushed, it should go back to the first step and start over.

\section*{Tag the release}

\noindent\graybox{git tag -a 2.90.6}

Note that the name of the tag will be the version of the release.

\section*{Push the tag}

\noindent\graybox{git push origin 2.90.6}

\newpage
\section*{Upload the tarball}

\GNOME provides a server where to upload the released tarball so other people can have access to them easily.

\noindent\graybox{scp gedit-2.90.6.tar.xz (user)@master.gnome.org:}

\section*{ssh into the server}

\noindent\graybox{ssh (user)@master.gnome.org}

\section*{Install the tarball in the server}

\noindent\graybox{ftpadmin install gedit-2.90.6.tar.xz}

Once installed, a confirmation should be received inedicating that it was correctly installed.
