% mainfile:main.tex

\chapter{libpeas}

A decision was made several years ago to try and convert the plugin system used in gedit into an standalone library. The main reason for this, was that several applications were copying the gedit plugin system and then making small modifications on it to adapt it as they preferred.  The main reason for not making this fix in prior versions was based upon maintaining the ABI stability for the production level code with a large user base.
As explained in the chapter \ref{chap:GFile}, the version update to \GNOME 3.0 provided an opportunity to effect this risky change.

With the help of the rest of the \emph{gedit team}, Steve Fr\'ecinaux was the one that took care of creating this library. Even if I am not the one who created this code,  I provided significant help both in the design and to port gedit (together with other applications), thereby 
indicating problems and providing some patches. This chapter provides an overview of the whole port, the changes in the design,  
and how it changed the implementation of plugins, pointing out examples and the advantages of the new design.

Note that one of the design decisions when libpeas was implemented was to make it gobject-introspection based (see \ref{sec:GObjectIntrospection}). This was one of the main problems, the port to use this library was not straightforward.

\section{Analysis and Design}

\subsection{Old Plugin system}\label{sec:OldPluginSystem}

\addfigure{./images/libpeas1}{Old Plugin system diagram}{fig:OldPluginSystemDiagram}

In Figure \ref{fig:OldPluginSystemDiagram}, the class diagram of the plugin system was checked before 
any work was done in the libpeas library. In order to better understand the next parts of this chapter, an overview of the 
main classes in this diagram is needed.

\subsubsection{GeditPluginsEngine}

This is the main class of the plugin engine. It takes care of loading and unloading a specific plugin, activate and deactivate the plugins for a specific window and to get the information for a specific plugin.

The GeditWindow has a direct link to this class by telling to it when it has to activate or deactivate a plugin. Also, it listens to the signals `activate\_plugin' and `deactivate\_plugin' (which should be called `load\_plugin' and `unload\_plugin') to actually activate them.

This class is a singleton so it can be accessed at any point by any class.

\newpage
\subsubsection{GeditPluginManager}

\addfigure[scale=0.5]{./images/gedit-plugins-dialog}{Gedit plugins dialog}{fig:GeditPluginsDialog}

In Figure \ref{fig:GeditPluginsDialog} the user interface is shown that  exposes gedit preferences dialog to be able 
to load or unload plugins. This widget calls to load or unload methods from GeditPluginsEngine. This later will emit 
the related signal listened by the window as spotted before.

\subsubsection{GeditPlugin}

This is the class implemented by each plugin. It has the methods that each plugin has to implement.

\subsubsection{Other classes}

The rest of the classes are mainly an implementation detail which are the ones in charge of loading a specific type of plugin, i.e Python vs C plugins.

\newpage
\subsection{First design with libpeas}

\addfigure[scale=0.5]{./images/libpeas2}{First design using libpeas}{fig:FirstDesignUsingLibpeas}

In the design given in Figure \ref{fig:FirstDesignUsingLibpeas}, the design is similar to the previous case. Libpeas 
takes the classes from \ref{sec:OldPluginSystem} and exposes the public ones for the user. We can see that we 
still have a GeditPluginsEngine, the reason for this is explained below.

\subsubsection{GeditPluginsEngine}

This is a lightweight object inherited from PeasEngine. PeasEngine is the same as the previous GeditPluginsEngine, although it does not 
initialize the paths where to search the plugins, the name of the plugins etc. Here is where it takes part the new GeditPluginsEngine, 
it initializes the super class by setting the name of the application and the plugin paths.

\subsection{Second design with libpeas}

\addfigure[scale=0.45]{./images/libpeas3}{Design using extensions}{fig:DesignExtensions}

One of the most frustrating limitations of the gedit plugins engine was that it only allows 
the extention of a single class, called GeditPlugin. With libpeas, this limitation vanishes, 
and the application writer is now able to provide a set of interfaces the plugin writer will be able to implement as the plugin requires.

In this design, we make use of the extension points. As we can see there are three extension points:
\begin{itemize}
  \item \textbf{GeditAppActivatable}: They are extensions that need to be instantiated per application instance. They will be accessible by all the windows. This is interesting if we want to read some specific data and create a singleton for it.
  \item \textbf{GeditWindowActivatable}: Extensions which interact directly with a window. For example the need to add a side pane or a bottom pane, menu items etc.
  \item \textbf{GeditViewActivatable}: Extensions which need either add new things to the document view, or to manage the document buffer to add new edition features.
\end{itemize}

For this, libpeas exposes a new class PeasExtensionSet. A PeasExtensionSet is a class that helps to manage all the specific extensions. As one plugin can adds one or more extensions, for example the window will have to deal with all the WindowActivatable extensions created by all the plugins. PeasExtensionSet ensures the life of each of them and allows the container class (GeditWindow) to send activations, deactivations or other calls needed to this extensions.

Apart from this new feature, the design is kept the same. The only problem is that it will produce a lot of work 
for porting the old plugins to this new design. In the next sections is shown an overview of how it had to be dealt 
with this in order to port the old plugins.

\newpage
\section{Old Plugin implementation}

Here a snippet is shown that indicates how the previous plugin looked:

\subsection{Minimalist plugin}

\begin{lstlisting}[style=python]

import gedit

class ExamplePyPlugin(gedit.Plugin):
    def activate(self, window):
        pass

    def deactivate(self, window):
        pass

    def update_ui(self, window):
        pass

\end{lstlisting}

As we can be seen in this example, each plugin must implement three methods:
\begin{enumerate}
  \item \textbf{activate:} called when the plugin is activated. This is usually used for the initialization of the plugin itself.
  \item \textbf{deactivate:} called when the plugin is deactivated. Used to dispose the plugin.
  \item \textbf{update\_ui:} called when a change in user interface of the window has been introduced. i.e when a tab in the window has been switched.
\end{enumerate}

\subsection{Window controller}\label{sec:WindowController}

As was explained in chapter \ref{chap:InitialAnalysis}, gedit is a \emph{single instance application}. This means that for each instance of the application we can have several windows. This is applied also to the plugins. One plugin instance will be created only once, and then the activate/deactivate/update\_ui methods will be called for each window.

This has forced us to introduce the \emph{window controller}, which is one class that takes care to control a specific window. The plugin instance will have a hash table that takes care of calling the right window controller instance method. This allows a simple way to implement plugins.

The main problem with this was that each plugin developer needed to know internal details of gedit in order to create a plugin correctly. In the example below, it can be shown how it was dealed with this before.

\begin{lstlisting}[style=python]

import gedit

class ExamplePyWindowHelper:
    def __init__(self, plugin, window):
        print "Plugin created for", window
        self._window = window
        self._plugin = plugin

    def deactivate(self):
        print "Plugin stopped for", self._window
        self._window = None
        self._plugin = None

    def update_ui(self):
        # Called whenever the window has been updated (active tab
        # changed, etc.)
        print "Plugin update for", self._window

class ExamplePyPlugin(gedit.Plugin):
    def __init__(self):
        gedit.Plugin.__init__(self)
        self._instances = {}

    def activate(self, window):
        self._instances[window] = ExamplePyWindowHelper(self, window)

    def deactivate(self, window):
        self._instances[window].deactivate()
        del self._instances[window]

    def update_ui(self, window):
        self._instances[window].update_ui()

\end{lstlisting}

\section{New plugin implementation}

A plugin will be able to have one or more extensions. Each extension is derived from GObject.Object and must implement one of the interfaces that gedit provides for the extension points.

\begin{lstlisting}[style=python]

from gi.repository import GObject, Gedit

class ExamplePyWindowActivatable:
    __gtype_name__ = "ExamplePyWindowActivatable"

    window = GObject.property(type=Gedit.Window)

    def __init__(self):
        GObject.Object.__init__(self)

    def do_activate(self):
        print "Plugin created for", self.window

    def do_deactivate(self):
        print "Plugin stopped for", self.window

    def do_update_state(self):
        # Called whenever the window has been updated (active tab
        # changed, etc.)
        print "Plugin update for", self.window

\end{lstlisting}

This example provides the same as the one provided in the section \ref{sec:WindowController}. As we can see, it is a lot easier to implement and it needs a smaller learning curve. For each window, a new extension point is created so the class ExamplePyWindowActivatable will be instantiated for each of them.

\section{Tests}

Unfortunately, to test this, we had to check all the functionalities. The main problem with this was that we had to port also the python plugin to use gobject-introspection. This meant that the methods did not take the same amount of arguments, and thus we needed to test all the code base and all the functionalities, one by one.

