% mainfile:main.tex
%  revised dnolivieri

\chapter{Development process}

% FIXME: draft

For each feature, understanding by feature, a bug in the application or one new functionality to be added:

**** I don't understand what you want to say ??? ****




\begin{itemize}
  \item Analysis
    \begin{itemize}
      \item Study of the problem.
      \item Discussion with the rest of developers.
      \item Create diagrams using reverse engineering if necessary.
      \item Analysis of the use cases unless it is not needed for the specific task if the feature is already made.
    \end{itemize}
  \item Design
    \begin{itemize}
      \item Study of the technologies to be used in the new feature.
      \item Study of the current design.
      \item Generate initial prototypes.
      \item Brainstorming with the rest of developers to check which could be the best design.
    \end{itemize}
  \item Codification
    \begin{itemize}
      \item Implementation using C with GObject or python (see \ref{sec:ProgrammingLanguages}), depending on the plugin or if it is a core feature.
      \item File a bug to review the patch.
    \end{itemize}
  \newpage
  \item Tests
    \begin{itemize}
      \item Test all new functionalities.
    \end{itemize}
  \item Deployment
    \begin{itemize}
      \item Commit and push the patches to the \GNOME repository.
    \end{itemize}
\end{itemize}

Note that in some of the features, the Analysis and the Design are correctly attached.   This is important so that it can be exposed in the documentation as a single explanation point.  The reason for this is that there are some problems like refactoring code, where all the requirements are already known, or the use of new technologies that may produce a reanalysis and a redesign to ensure that the program will continue having the right use cases.


